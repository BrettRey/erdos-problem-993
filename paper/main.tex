% !TEX TS-program = xelatex
\documentclass[12pt,oneside]{article}
\input{preamble.tex}

% ============================
% Additional packages
% ============================
\usepackage{amsthm}

% Theorem environments
\newtheorem{theorem}{Theorem}
\newtheorem{lemma}{Lemma}
\newtheorem{corollary}{Corollary}
\newtheorem{conjecture}{Conjecture}
\newtheorem{definition}{Definition}
\theoremstyle{remark}
\newtheorem{remark}{Remark}

% ============================
% Notation macros
% ============================
\newcommand{\indpoly}{I(T;\,x)}           % independence polynomial
\newcommand{\ik}{i_k(T)}                  % coefficient
\newcommand{\nm}{\mathrm{nm}}             % near-miss ratio
\newcommand{\broom}{\mathrm{B}}           % broom notation

% ============================
% Metadata
% ============================
\title{Unimodality of independence polynomials of trees:\\
  computational verification and extremal asymptotics}
\author{Brett Reynolds\,\orcidlink{0000-0003-2407-9448}}
\date{}

\begin{document}
\maketitle

\begin{abstract}
\citet{alavi1987} conjectured that the independence polynomial of every tree is unimodal.
I report computational results bearing on this conjecture and an elementary analytic result for leaf-attachment asymptotics.
First, I verify the conjecture exhaustively for all 447\,672\,596 non-isomorphic trees on at most 26 vertices, extending the previously published bound of $n \le 25$.
At $n = 26$ I confirm exactly two log-concavity failures and no unimodality failures, with every tree's coefficient sequence comfortably unimodal.
Second, I test 145\,362 trees from five structured families (subdivided stars, caterpillars, spiders, brooms, and random perturbations) at sizes up to $n = 500$, finding zero unimodality violations.
Third, an evolutionary search over tree space identifies \emph{multi-arm stars}~-- stars with multiple short paths emanating from the hub~-- as the family approaching the unimodality boundary most closely, surpassing the brooms previously believed to be extremal.
The near-miss ratio satisfies $\nm(s) = 1 - C/s + O(1/s^2)$ with $C \approx 4.0$ for the best multi-arm configurations (vs.\ $C \approx 4.1$ for brooms), so the margin of unimodality shrinks as $O(1/s)$ but never vanishes.
The same asymptotic mechanism applies to any fixed core with $s$ pendant leaves at a single vertex.
I also give an elementary proof that brooms are unimodal whenever the star size dominates the path ($s \ge p$).
Finally, I develop a reduction framework: via a Hub Exclusion Lemma and a Transfer Lemma, unimodality for all trees reduces to a single conjecture about trees in which every vertex has at most one leaf-child, verified computationally for over 8.9 million trees through $n = 22$.
\end{abstract}

\section{Introduction}\label{sec:intro}

An \emph{independent set} in a graph $G$ is a set of vertices no two of which are adjacent.
Write $i_k(G)$ for the number of independent sets of size~$k$ in~$G$, and let $\alpha(G)$ denote the independence number (the maximum size of an independent set).
The \emph{independence polynomial} of~$G$ is
\[
  I(G;\, x) \;=\; \sum_{k=0}^{\alpha(G)} i_k(G)\, x^k,
\]
with the convention $i_0(G) = 1$.

\begin{conjecture}[\citealp{alavi1987}]\label{conj:main}
  For every tree~$T$, the sequence $i_0(T), i_1(T), \dotsc, i_{\alpha(T)}(T)$ is unimodal.
\end{conjecture}

\citet{alavi1987} showed that the independence sequence of a general graph can realize any prescribed shape, but conjectured that trees are constrained enough to force unimodality.
The conjecture naturally suggested a stronger form: that the sequence should be \emph{log-concave}, meaning $i_k^2 \ge i_{k-1}\, i_{k+1}$ for all~$k$.
This stronger property held for all trees tested computationally until $n = 26$, when \citet{kadrawi2023} found exactly two trees whose independence sequences are unimodal but not log-concave.
\citet{galvin2025} subsequently constructed infinite families of trees (subdivided stars) with log-concavity failures arbitrarily far from the ends of the sequence.
\citet{ramos2025} used machine learning (PatternBoost) to find tens of thousands of log-concavity failures up to $n = 101$, but no unimodality failure.

On the positive side, unimodality has been proved for several tree families: paths, centipedes, regular caterpillars, and Fibonacci trees (which are real-rooted, hence log-concave), as well as well-covered spiders and certain restricted caterpillars.
\citet{levit2006} showed that $i_k(T)$ is strictly decreasing for $k \ge \lceil (2\alpha - 1)/3 \rceil$, so any unimodality violation must occur in roughly the first two-thirds of the sequence.
\citet{heilman2025} proved that for a uniformly random labelled tree on $n$ vertices, the first $\approx 46.8\%$ of the independence sequence is increasing with high probability; combined with the Levit--Mandrescu tail bound, roughly 85\% of the sequence is known to behave as expected almost surely.

The best previously published exhaustive verification is $n \le 25$ (A.~Radcliffe, personal communication, cited in several papers).
I extend this to $n \le 26$, test structured families up to $n = 500$, and identify multi-arm stars as the trees closest to violating unimodality.

My contributions:
\begin{enumerate}
  \item Exhaustive verification of Conjecture~\ref{conj:main} for all trees on $n \le 26$ vertices (Section~\ref{sec:exhaustive}).
  \item Targeted verification across five tree families at sizes up to $n = 500$ (Section~\ref{sec:targeted}).
  \item Identification of multi-arm stars as the extremal family for the near-miss ratio, discovered via evolutionary optimization over tree space (Section~\ref{sec:targeted}).
  \item An empirical asymptotic analysis showing that leaf-attachment trees approach the unimodality boundary at rate $O(1/s)$ (Section~\ref{sec:broom}).
  \item An elementary proof that brooms are unimodal whenever $s \ge p$ (Section~\ref{sec:broom}).
  \item A reduction framework~-- via Hub Exclusion and Transfer Lemmas~-- showing that unimodality for all trees reduces to a single conjecture about $d_{\mathrm{leaf}} \le 1$ trees, verified for over 8.9 million trees through $n = 22$ (Section~\ref{sec:discussion}).
\end{enumerate}
\begin{remark}
  Li, Li, Yang, and Zhang~\citep{li2025spiders} have proved that all spiders are strongly log-concave, which implies log-concavity (hence unimodality) for brooms as a special case. I include the elementary $s \ge p$ proof below as a self-contained argument that aligns with the asymptotic analysis in Section~\ref{sec:broom}.
\end{remark}
These results should be read as four complementary evidence types: exhaustive verification provides certified confirmation within the tested range; targeted family searches are adversarial stress-tests; evolutionary optimization identifies extremal structure; and the leaf-attachment analysis offers an explanatory asymptotic model rather than a general proof.


\section{Definitions and method}\label{sec:method}

\begin{definition}
  A finite sequence $(a_0, a_1, \dotsc, a_m)$ of positive reals is \emph{unimodal} if there exists an index~$p$ such that $a_0 \le a_1 \le \dotsb \le a_p \ge a_{p+1} \ge \dotsb \ge a_m$.
  It is \emph{log-concave} if $a_k^2 \ge a_{k-1}\, a_{k+1}$ for all $1 \le k \le m-1$.
\end{definition}

Log-concavity (with positive terms) implies unimodality, but not conversely.

\paragraph*{Near-miss ratio.}
To quantify how close a tree comes to violating unimodality, I define the \emph{near-miss ratio}.
Let $j_0$ be the first index where $i_{j_0} > i_{j_0+1}$ (the first strict descent).
The near-miss ratio is
\[
  \nm(T) \;=\; \max_{j > j_0} \frac{i_{j+1}(T)}{i_j(T)}.
\]
If $\nm(T) > 1$, the sequence rises after descending, violating unimodality.
A value near 1 indicates that the tree nearly violates it.
Thus $\nm$ serves as a quantitative measure of how tight the conjecture is for a family.

\paragraph*{Tree DP.}
The computation of $I(T;\, x)$ proceeds by rooting $T$ at an arbitrary vertex and performing a bottom-up traversal.
For each vertex~$v$, two polynomials are maintained: $d_v^{(0)}(x)$ counts independent sets in the subtree of~$v$ that exclude~$v$; $d_v^{(1)}(x)$ counts those that include~$v$.
The recurrences are
\begin{align}
  d_v^{(0)} &= \prod_{c \,\in\, \mathrm{children}(v)} \bigl(d_c^{(0)} + d_c^{(1)}\bigr), \label{eq:dp0}\\
  d_v^{(1)} &= x \cdot \prod_{c \,\in\, \mathrm{children}(v)} d_c^{(0)}, \label{eq:dp1}
\end{align}
and $I(T;\, x) = d_r^{(0)} + d_r^{(1)}$ for the root~$r$.
The total time per tree is $O(n^2)$, dominated by polynomial multiplications via naive convolution.

\paragraph*{Computational setup.}
Trees are enumerated with \texttt{nauty/geng} \citep{mckay2014}, which generates all non-isomorphic connected graphs on $n$ vertices with exactly $n - 1$ edges.
Computation is parallelized by passing \texttt{geng}'s \texttt{res/mod} partitioning to eight workers via Python's \texttt{multiprocessing}.
Polynomial arithmetic uses \texttt{numpy.convolve} when coefficients fit in 64-bit integers, with automatic fallback to pure Python big-integer arithmetic for larger values (relevant at $n \gtrsim 90$).
Log-concavity is checked using integer arithmetic ($i_k^2$ vs.\ $i_{k-1}\, i_{k+1}$), avoiding floating-point error.


\section{Exhaustive verification through $n = 26$}\label{sec:exhaustive}

Table~\ref{tab:exhaustive} shows the verification results.
Every tree count matches OEIS A000055 \citep{oeis-a000055}.
No unimodality violation was found at any~$n$.
Counts are checked automatically against A000055 during the search; the run halts on any mismatch.

\begin{table}[ht]
  \centering
  \caption{Exhaustive verification of unimodality for trees on $n$ vertices.  Times are for an Apple Silicon Mac with 8 parallel workers ($n \ge 21$) or a single process ($n \le 20$).}
  \label{tab:exhaustive}
  \begin{tabular}{r r r}
    \toprule
    $n$ & Trees & Time \\
    \midrule
    1--15 & \liningnums{13\,188} & ${<}\,1$\,s \\
    16 & \liningnums{19\,320} & 1\,s \\
    17 & \liningnums{48\,629} & 3\,s \\
    18 & \liningnums{123\,867} & 9\,s \\
    19 & \liningnums{317\,955} & 23\,s \\
    20 & \liningnums{823\,065} & 68\,s \\
    21 & \liningnums{2\,144\,505} & 55\,s \\
    22 & \liningnums{5\,623\,756} & 1\,m\,44\,s \\
    23 & \liningnums{14\,828\,074} & 4\,m\,41\,s \\
    24 & \liningnums{39\,299\,897} & 12\,m\,5\,s \\
    25 & \liningnums{104\,636\,890} & 38\,m\,33\,s \\
    26 & \liningnums{279\,793\,450} & 4\,h\,51\,m \\
    \midrule
    Total & \liningnums{447\,672\,596} & \\
    \bottomrule
  \end{tabular}
\end{table}

The case $n = 26$ is particularly interesting because it is the first~$n$ at which log-concavity fails \citep{kadrawi2023}.
A re-analysis of all 279\,793\,450 trees at $n = 26$ found:
\begin{itemize}
  \item Exactly 2 log-concavity failures (consistent with \citealp{kadrawi2023}), both at index $k = 13$.
    The worst log-concavity ratio is $i_{12}\, i_{14} / i_{13}^2 = 1.145$, exceeding the threshold of~1.
  \item The best (highest) near-miss ratio at $n = 26$ is $\nm = 0.845$, far below the violation threshold of~1.
\end{itemize}

In other words, even at the first~$n$ where the stronger log-concavity property breaks, unimodality holds with a comfortable margin: the tightest near-miss is only 84.5\% of the way to a violation.


\section{Targeted search on structured families}\label{sec:targeted}

Beyond exhaustive enumeration, I tested 145\,362 trees from five families chosen for their proximity to the unimodality boundary.
Results appear in Table~\ref{tab:targeted}.
This is an adversarial stress-test rather than a proof: the absence of violations only rules out counterexamples within these families and size ranges.

\begin{table}[ht]
  \centering
  \caption{Targeted verification on structured tree families ($n$ up to 500).
    LC = log-concavity.  Best~nm is the highest near-miss ratio observed in the family.}
  \label{tab:targeted}
  \begin{tabular}{l r r l}
    \toprule
    Family & Trees & LC failures & Best nm \\
    \midrule
    Galvin SST $T_{m,t,1}$     & \liningnums{571}     & \liningnums{108} & 0.936 \\
    Generalized SST $T_{m,t,d}$ & \liningnums{680}    & \liningnums{268} & 0.981 \\
    Caterpillars                & \liningnums{5\,196}  & 0 & -- \\
    Spiders and brooms          & \liningnums{133\,915}& 0 & 0.992 \\
    Random (Ramos--Sun style)   & \liningnums{5\,000}  & \liningnums{2} & 0.804 \\
    \midrule
    Total                       & \liningnums{145\,362}& \liningnums{378} &  \\
    \bottomrule
  \end{tabular}
\end{table}

A complementary structural scan considered the class
\[
  \mathcal{C}_2 = \{T:\ |\{v \in V(T): \deg(v)\ge 3\}| \le 2\},
\]
which strictly contains spiders.
Exhaustive filtering of all non-isomorphic trees through $n=24$ gave 196\,635 trees in $\mathcal{C}_2$ and found zero log-concavity failures (hence zero unimodality failures).
Equivalently: all trees with at most two branch vertices up to $n=24$ are log-concave (hence unimodal).
The worst observed log-concavity ratio was $0.846153846\ldots$, still well below the violation threshold of~1.
Full run details are archived in \texttt{results/two\_branch\_lc\_n24.json}.

\begin{conjecture}\label{conj:c2}
  Every tree in~$\mathcal{C}_2$ (at most two branch vertices) is log-concave.
\end{conjecture}

\begin{table}[ht]
  \centering
  \caption{Compact summary of the $\mathcal{C}_2$ scan for $n=20,\dots,24$.}
  \label{tab:c2scan}
  \begin{tabular}{r r r r}
    \toprule
    $n$ & Trees & $\mathcal{C}_2$ trees & LC failures in $\mathcal{C}_2$ \\
    \midrule
    20 & \liningnums{823\,065}    & \liningnums{13\,463} & 0 \\
    21 & \liningnums{2\,144\,505} & \liningnums{20\,267} & 0 \\
    22 & \liningnums{5\,623\,756} & \liningnums{30\,170} & 0 \\
    23 & \liningnums{14\,828\,074}& \liningnums{44\,385} & 0 \\
    24 & \liningnums{39\,299\,897}& \liningnums{64\,674} & 0 \\
    \bottomrule
  \end{tabular}
\end{table}

No unimodality violation was found in any family.
Three patterns stand out:
\begin{itemize}
  \item \textbf{Subdivided stars dominate log-concavity failures} (376 of 378), consistent with \citet{galvin2025}, but their near-miss ratios stay well below~1.
  \item \textbf{Caterpillars are perfectly log-concave} through $n = 500$, with zero failures.
  \item \textbf{Brooms produce higher near-miss ratios than all SST variants} ($\nm = 0.991$ at $n = 500$) despite having no log-concavity failures.
\end{itemize}

A \emph{broom} $\broom(p, s)$ consists of a path on $p$ vertices with a star of $s$ leaves attached at one end; it has $n = p + s$ vertices and independence number $\alpha = \lfloor p/2 \rfloor + s$.
However, brooms are not the true extremal family.

\paragraph*{Multi-arm stars.}
To search for trees that maximize $\nm$, I ran an evolutionary optimizer over tree space using mutations (leaf relocation, subtree prune-and-regraft, pendant concentration) with elite selection.
The optimizer consistently converged to a generalization of brooms that I call \emph{multi-arm stars}.

\begin{definition}[Multi-arm star]
  For integers $s \ge 0$ and $k \ge 1$ with arm lengths $a_1 \ge a_2 \ge \cdots \ge a_k \ge 1$, the \emph{multi-arm star} $M(s;\, a_1, \dotsc, a_k)$ is the tree with a central hub vertex~$h$, $s$~pendant leaves adjacent to~$h$, and $k$~paths of lengths $a_1, \dotsc, a_k$ emanating from~$h$.  The total vertex count is $n = 1 + s + \sum a_i$.
  A standard broom $\broom(p,s)$ is the special case $M(s;\, p-1)$ with $k = 1$.
\end{definition}

Table~\ref{tab:multiarm} compares the best standard broom against the best multi-arm star at each vertex count.
Multi-arm stars achieve higher $\nm$ at every~$n$ tested, with the optimal arm configuration transitioning from two arms at small~$n$ to four arms at $n \ge 200$.
\begin{table}[ht]
  \centering
  \caption{Best standard broom vs.\ best multi-arm star at each~$n$.
    All values are near-miss ratios; higher means closer to violation.}
  \label{tab:multiarm}
  \begin{tabular}{r l l l}
    \toprule
    $n$ & Best broom $\nm$ & Best multi-arm $\nm$ & Configuration \\
    \midrule
    75   & 0.9412 & 0.9437 & 2-arm$(6,2)$ \\
    100  & 0.9551 & 0.9575 & 3-arm$(6,3,2)$ \\
    200  & 0.9779 & 0.9792 & 4-arm$(5,5,4,2)$ \\
    500  & 0.9913 & 0.9918 & 4-arm$(5,5,4,2)$ \\
    1000 & 0.9957 & 0.9959 & 4-arm$(5,5,4,2)$ \\
    \bottomrule
  \end{tabular}
\end{table}

The champion at $n \ge 200$ is the 4-arm star with arms of lengths 5, 5, 4, and~2.
At $n = 500$, this configuration achieves $\nm = 0.9918$, modestly but consistently above the best broom's $\nm = 0.9913$.
The gap narrows as $n$ grows (both families satisfy $\nm \to 1$ at rate $O(1/s)$), but the ordering is stable across all sizes tested.
This makes multi-arm stars, not brooms, the family closest to the unimodality boundary.


\section{Broom asymptotics}\label{sec:broom}

The targeted results raised a natural question: does the near-miss ratio converge to~1 as $s \to \infty$?
If so, the conjecture would be true for these families but only barely, with the margin vanishing in the limit.
I analyse the asymptotic mechanism using brooms as the simplest case, then show it applies to any fixed core (including multi-arm stars) with $s$ pendant leaves at a single vertex.

\begin{theorem}[Broom unimodality for large stars]\label{thm:broom-sgeqp}
Fix $p \ge 2$. For all $s \ge p$, the broom $\broom(p,s)$ is unimodal.
\end{theorem}

\begin{proof}
Write $A(x) = I(P_{p-1}; x) = \sum_t a_t x^t$ and $B(x) = I(P_{p-2}; x) = \sum_t b_t x^t$.
Then
\[
  I(\broom(p,s); x) = (1+x)^s A(x) + x B(x).
\]
Let $c_k$ be the coefficient of $x^k$ in $I(\broom(p,s); x)$ and write $c_k = d_k + e_k$, where
\[
  d_k = \sum_{j=0}^{\min(s,k)} \binom{s}{j}\, a_{k-j}, \qquad e_k = b_{k-1}.
\]
Let $S = \lfloor (p-1)/2 \rfloor + 1$, so $e_k = 0$ for $k \ge S+1$.
Since $a_t$ and $\binom{s}{j}$ are log-concave with no internal zeros, $d_k$ is log-concave and hence unimodal.
It suffices to show $c_{k+1} \ge c_k$ for all $0 \le k \le S-1$.

For such $k$, change variables $t = k+1-j$ to obtain
\[
  d_{k+1} - d_k = \sum_t a_t\Bigl(\binom{s}{k+1-t} - \binom{s}{k-t}\Bigr).
\]
Because $k+1 \le S = \lceil p/2 \rceil \le \lceil s/2 \rceil$, and the binomial row is nondecreasing up to $r = \lceil s/2 \rceil$,
each term in parentheses is nonnegative.
Keeping only the $t=1$ term gives
\[
  d_{k+1} - d_k \ge a_1\Bigl(\binom{s}{k} - \binom{s}{k-1}\Bigr) = (p-1)\Bigl(\binom{s}{k} - \binom{s}{k-1}\Bigr).
\]
For fixed $k \le \lfloor p/2 \rfloor$, the function $s \mapsto \binom{s}{k} - \binom{s}{k-1}$ is increasing for $s \ge 2k-1$, hence
\[
  \binom{s}{k} - \binom{s}{k-1} \ge \binom{p}{k} - \binom{p}{k-1}.
\]
Using $\binom{p}{k} = \binom{p}{k-1}\frac{p-k+1}{k}$, this implies
\[
  d_{k+1} - d_k \ge (p-1)\binom{p}{k-1}\frac{p-2k+1}{k}.
\]
Since $k \le \lfloor p/2 \rfloor$, we have $p-2k+1 \ge 1$ and thus
$(p-1)\binom{p}{k-1}\frac{p-2k+1}{k} \ge \binom{p}{k-1}$.
Finally, $e_k - e_{k+1} = b_{k-1} - b_k \le b_{k-1} = \binom{p-k}{k-1} \le \binom{p}{k-1}$.
Therefore $d_{k+1} - d_k \ge e_k - e_{k+1}$, so $c_{k+1} \ge c_k$ for $0 \le k \le S-1$.
For $k=S$ (the even-$p$ boundary case), the same bound gives
$d_{S+1}-d_S \ge \binom{p}{S-1} \ge b_{S-1}=e_S$, hence $c_{S+1}\ge c_S$.

For $k \ge S+1$, we have $c_k = d_k$, which is unimodal. Hence $c_k$ is unimodal for all $s \ge p$.
\end{proof}

I computed $\broom(13, s)$ for $s$ up to 20\,000 (trees with over 20\,000 vertices, independence polynomials with degrees in the tens of thousands).
Table~\ref{tab:broom} shows the results.

\begin{table}[ht]
  \centering
  \caption{Broom $\broom(13, s)$: convergence of the near-miss ratio.
    The scaled gap $s \cdot (1 - \nm)$ stabilizes near $C \approx 4.12$.}
  \label{tab:broom}
  \begin{tabular}{r r l l l}
    \toprule
    $s$ & $n$ & $\nm$ & $1 - \nm$ & $s \cdot (1 - \nm)$ \\
    \midrule
    1\,000  & 1\,013  & 0.995\,911 & 0.004\,09 & 4.089 \\
    2\,000  & 2\,013  & 0.997\,947 & 0.002\,05 & 4.105 \\
    5\,000  & 5\,013  & 0.999\,177 & 0.000\,82 & 4.115 \\
    10\,000 & 10\,013 & 0.999\,588 & 0.000\,41 & 4.119 \\
    20\,000 & 20\,013 & 0.999\,794 & 0.000\,21 & 4.120 \\
    \bottomrule
  \end{tabular}
\end{table}

The data fit the scaling law
\begin{equation}\label{eq:scaling}
  \nm(s) \;=\; 1 - \frac{C}{s} + O(1/s^2),
\end{equation}
with $C \approx 4.12$ for $p = 13$.
The same qualitative behaviour was observed (convergence of $s \cdot (1 - \nm)$ to a constant) for path lengths $p \in \{13, 22, 33, 42, 50\}$, with the constant~$C$ varying by path length.
In every case, $\nm < 1$ for all tested~$s$, so all brooms remain unimodal.

\paragraph*{Heuristic for the $1 - C/s$ law.}
More generally, let $H$ be any fixed tree with distinguished vertex $v$, and let $H_s$ be obtained
by attaching $s$ new leaves to $v$. The vertex-deletion recurrence gives
\[
  I(H_s; x) = (1+x)^s\, A(x) + x\, B(x),
  \quad A(x) = I(H - v; x), \quad B(x) = I(H - N[v]; x).
\]

For brooms, let $\broom(p, s)$ be a path on $p$ vertices with $s$ leaves attached at one end.
Applying the same recurrence at the attachment vertex gives the closed form
\begin{equation}\label{eq:broom-closed}
  I(\broom(p, s); x) \;=\; (1+x)^s\, I(P_{p-1}; x) \;+\; x\, I(P_{p-2}; x),
\end{equation}
where $P_m$ is the path on $m$ vertices.
Write $A(x) = I(P_{p-1}; x) = \sum_{j=0}^{d} a_j x^j$.
The quantities $A(1)$ and $A'(1)$ that govern the scaling constant are computable from a simple recurrence.
Since $I(P_n; x) = I(P_{n-1}; x) + x\, I(P_{n-2}; x)$, setting $F_n = I(P_n; 1)$ and $G_n = I'(P_n; 1)$ gives
\begin{align}
  F_n &= F_{n-1} + F_{n-2}, \label{eq:Fn}\\
  G_n &= G_{n-1} + G_{n-2} + F_{n-2}, \label{eq:Gn}
\end{align}
with $F_0 = 1$, $F_1 = 2$, $G_0 = 0$, $G_1 = 1$.
In particular, $F_n$ is the $(n+2)$-nd Fibonacci number, and $A(1) = F_{p-1}$, $A'(1) = G_{p-1}$.

For large $s$, the $x\, I(P_{p-2}; x)$ term in~\eqref{eq:broom-closed} is negligible in the central coefficient window: its contribution to $c_k$ is bounded by $I(P_{p-2}; 1)$, while the leading term contributes $\Theta(\binom{s}{k}) \cdot A(1)$, which is exponentially larger near $k \approx s/2$.
The omission perturbs $r_k = c_{k+1}/c_k$ only at $O(1/s^2)$.

Dropping this term, for $k \approx s/2$ we have
\[
  c_k \;\approx\; \sum_{j=0}^{d} a_j \binom{s}{k-j}.
\]
For fixed $j$ and $k = s/2 + y$ with $y = O(1)$,
\[
  \frac{\binom{s}{k+1-j}}{\binom{s}{k-j}}
  \;=\; \frac{s-k+j}{k+1-j}
  \;=\; 1 - \frac{4y + 2 - 4j}{s} + O(s^{-2}).
\]
Taking the $a_j$-weighted average yields
\[
  r_k \;=\; 1 - \frac{4y + 2 - 4\mu}{s} + O(s^{-2}),
  \quad \mu = \frac{A'(1)}{A(1)} = \frac{G_{p-1}}{F_{p-1}}.
\]
The first descent occurs near $y_0 \approx \mu - \tfrac{1}{2}$.
Because the near-miss ratio scans indices strictly after the first descent, the
maximal ratio occurs at the smallest integer $m \ge \mu + \tfrac{1}{2}$, giving
\begin{equation}\label{eq:Cp}
  \nm(s) \;\approx\; 1 - \frac{C(p)}{s}, \quad
  C(p) = (4m + 2) - 4\mu.
\end{equation}
Table~\ref{tab:Cpred} compares the predicted $C(p)$ from~\eqref{eq:Cp} against the empirical value at $s = 5{,}000$ for each tested path length.

\begin{table}[ht]
  \centering
  \caption{Predicted vs.\ observed scaling constant $C(p)$ for broom $\broom(p, s)$.
    $C_{\mathrm{pred}}$ is computed from the Fibonacci recurrences~\eqref{eq:Fn}--\eqref{eq:Gn};
    $C_{\mathrm{obs}}$ is $s \cdot (1 - \nm)$ at $s = 5{,}000$.}
  \label{tab:Cpred}
  \begin{tabular}{r r r l l l}
    \toprule
    $p$ & $F_{p-1}$ & $G_{p-1}$ & $\mu$ & $C_{\mathrm{pred}}$ & $C_{\mathrm{obs}}$ \\
    \midrule
    13 & \liningnums{377} & \liningnums{1\,308} & 3.469 & 4.122 & 4.115 \\
    22 & \liningnums{28\,657} & \liningnums{170\,711} & 5.957 & 6.172 & 6.156 \\
    33 & \liningnums{5\,702\,887} & \liningnums{51\,310\,978} & 8.997 & 6.011 & 5.991 \\
    42 & \liningnums{433\,494\,437} & \liningnums{4\,978\,643\,596} & 11.485 & 4.060 & 4.045 \\
    50 & \liningnums{20\,365\,011\,074} & \liningnums{278\,920\,277\,425} & 13.696 & 7.216 & 7.182 \\
    \bottomrule
  \end{tabular}
\end{table}

The agreement is within 0.5\% in all cases, confirming that the heuristic captures the leading behaviour.

\begin{theorem}[Asymptotic leaf-attachment unimodality]\label{thm:leaf-attach-asymptotic}
Let $H$ be a fixed tree with distinguished vertex $v$, and let $H_s$ be obtained by attaching
$s$ new leaves to $v$. Write $A(x) = I(H - v; x)$ with
$\mu = A'(1)/A(1)$ and $m = \lceil \mu + \tfrac{1}{2} \rceil$.
Let $\nm(s)$ be the near-miss ratio of $H_s$.
Then, as $s \to \infty$,
\[
  \nm(s) \;=\; 1 - \frac{C}{s} + O\!\left(\frac{1}{s^2}\right),
  \quad C = (4m + 2) - 4\mu.
\]
In particular, $C \in [4, 8)$, so $\nm(s) < 1$ for all sufficiently large $s$ and $H_s$ is unimodal for all $s \ge s_0(H)$.
\end{theorem}

\begin{proof}
Let $B(x) = I(H - N[v]; x)$ and write
$I(H_s; x) = (1+x)^s A(x) + x B(x)$.
Let $c_k = [x^k] I(H_s; x)$ and write $c_k = d_k + e_k$ with
\[
  d_k = \sum_{j=0}^{d} a_j \binom{s}{k-j}, \qquad
  e_k = [x^k]\, x B(x),
\]
where $A(x)=\sum_{j=0}^d a_j x^j$ and $d = \deg A$ depends only on $H$.
Define $r_k = d_{k+1}/d_k$.
Since
\[
  r_k
  = \frac{\sum_{j=0}^d a_j \binom{s}{k-j} \,\frac{\binom{s}{k+1-j}}{\binom{s}{k-j}}}
         {\sum_{j=0}^d a_j \binom{s}{k-j}},
\]
$r_k$ is a weighted average of the ratios
$\rho_j(k) = \binom{s}{k+1-j}/\binom{s}{k-j} = (s-k+j)/(k+1-j)$.
As $j$ increases, $\rho_j(k)$ increases, so $r_k \in [\rho_0(k), \rho_d(k)]$.
In particular, for $k \le \lfloor s/2 \rfloor - d - 1$ we have $\rho_0(k) > 1$, hence $r_k > 1$,
and for $k \ge \lfloor s/2 \rfloor + d$ we have $\rho_d(k) < 1$, hence $r_k < 1$.
Thus the first descent occurs for $k = \lfloor s/2 \rfloor + y$ with $|y| \le d+1$.

For such $k$, each $\rho_j(k)$ admits the uniform expansion
\[
  \rho_j(k) = 1 - \frac{4y + 2 - 4j}{s} + O\!\left(\frac{1}{s^2}\right),
\]
since $j$ and $y$ are $O_H(1)$. Taking the weighted average yields
\[
  r_k = 1 - \frac{4y + 2 - 4\mu}{s} + O\!\left(\frac{1}{s^2}\right),
  \qquad \mu = \frac{\sum_j j a_j}{\sum_j a_j} = \frac{A'(1)}{A(1)}.
\]
Moreover, in this central window we have $e_k/d_k = O(\sqrt{s}\,2^{-s})$ because
$e_k$ is $O_H(1)$ while $d_k \asymp \binom{s}{\lfloor s/2 \rfloor}$, so
$c_{k+1}/c_k = r_k + O(\sqrt{s}\,2^{-s})$.
Therefore $c_{k+1}/c_k$ decreases by $4/s + O(1/s^2)$ when $k$ increments by~1,
and the first descent occurs at $y_0 = \mu - \tfrac{1}{2} + O(1/s)$.
The maximal ratio after the descent is attained at the smallest integer
$m \ge \mu + \tfrac{1}{2}$, giving
\[
  \nm(s) = 1 - \frac{(4m + 2) - 4\mu}{s} + O\!\left(\frac{1}{s^2}\right).
\]
Since $m - (\mu + \tfrac{1}{2}) \in [0, 1)$, we have $C \in [4, 8)$, so
$\nm(s) < 1$ for all sufficiently large $s$.
\end{proof}

\begin{lemma}[Leaf-attachment boundary negativity]\label{lem:leaf-attach-boundary}
Let $I_s(x) = (1+x)^s A(x) + x B(x)$ with $A(x)=\sum_{j=0}^d a_j x^j$,
$B(x)=\sum_{j=0}^e b_j x^j$, and $a_j,b_j \ge 0$.
Assume $s \ge e+1-d$ so $\alpha_s=\deg I_s = s+d$, and set
$t_s=\lceil(2\alpha_s-1)/3\rceil$.
If
\[
  s \;\ge\; s_0(d,e) \;:=\; \max\!\left(e+1-d,\; 2d+11,\; \Big\lceil\frac{3e-2d+8}{2}\Big\rceil\right),
\]
then
\[
  \Delta I_s{}_{t_s-2} \le 0
  \quad\text{and}\quad
  \Delta I_s{}_{t_s-1} \le 0.
\]
\end{lemma}

\begin{proof}[Proof sketch]
Write $U_s=(1+x)^sA$ and $V=xB$. The ratio envelope
$U_{s,k+1}/U_{s,k} \le (s-k+d)/(k+1-d)$ implies $\Delta U_{s,k}\le 0$ once
$k\ge (s+2d-1)/2$.
The condition $s\ge 2d+11$ places $k=t_s-2,t_s-1$ to the right of this threshold.
The condition $s\ge \lceil(3e-2d+8)/2\rceil$ ensures $t_s-2\ge e+1$, so
$\Delta V_{t_s-2},\Delta V_{t_s-1}\le 0$ (indeed $=0$ except possibly at $e+1$).
Thus $\Delta I_s=\Delta U_s+\Delta V$ is nonpositive at both boundary indices.
\end{proof}

\begin{remark}
  The constant $C$ depends only on $\mu$ and its ceiling. For brooms,
  $\mu = G_{p-1}/F_{p-1}$ by the Fibonacci recurrences~\eqref{eq:Fn}--\eqref{eq:Gn},
  so $C(p)$ oscillates between 4 and 8 as $p$ varies.
  Multi-arm stars have a different core $H$ and hence different $\mu$; the champion 4-arm$(5,5,4,2)$
  achieves $C \approx 3.98$, slightly below the broom optimum of $C \approx 4.12$.
  The theorem gives eventual unimodality for any fixed core; it does not
  address small $s$. Lemma~\ref{lem:leaf-attach-boundary} shows that once the
  leaf attachment is large enough relative to the core, the boundary indices are
  already nonincreasing. For brooms, log-concavity (hence unimodality) for all
  $s$ follows from the spider theorem of Li et al.\ \citep{li2025spiders}.
\end{remark}


\section{Discussion}\label{sec:discussion}

These results provide strong computational evidence for Conjecture~\ref{conj:main}.
Across 447.7 million exhaustively tested trees and 145 thousand structurally targeted trees, no unimodality violation was found.
The conjecture appears robust, but the asymptotic analysis reveals that it is also tight: the margin of unimodality can be made arbitrarily small.
Read as evidence types, the exhaustive portion certifies the conjecture within the tested range, the targeted families act as adversarial probes, the evolutionary optimizer identifies extremal structure, and the leaf-attachment analysis offers an explanatory asymptotic model for near-boundary behavior.
The fixed-core leaf-attachment theorem shows that the $1/s$ near-miss scaling is a generic effect of binomial smoothing, applying equally to brooms and multi-arm stars.
As a finite-core stress test motivated by the leaf-heavy reduction, I enumerated all degree-2-free trees obtained by attaching at most $\lambda_0$ leaves to each core vertex for several bounds.
No non-unimodal example appears for $(b_0,\lambda_0)=(8,4)$ (711{,}191 candidates) or $(7,5)$ (513{,}699 candidates), suggesting that any minimal obstruction must be very small if it exists.

Two structural surprises emerge.
First, subdivided stars~-- despite producing abundant log-concavity failures~-- stay well within the unimodality threshold.
The mechanisms behind log-concavity failure and near-unimodality violation are distinct.
Second, distributing the path structure across several short arms (multi-arm stars) pushes the near-miss ratio closer to~1 than concentrating it in a single long path (brooms).
The optimal arm configuration changes with~$n$: two arms at small~$n$, transitioning to three or four arms by $n = 200$.
Intuitively, shorter balanced arms preserve more pendant leaves for the same~$n$, and the resulting perturbation of the independence polynomial produces more effective near-miss amplification.

\subsection*{Toward a proof: a single-conjecture reduction}

The computational evidence motivates a search for a proof.
I develop a reduction showing that the conjecture follows from a single structural property of trees.

Call a maximal independent set $S$ \emph{1-Private} if every $u \in S$ has at most one private neighbor (a vertex $v \notin S$ whose only $S$-neighbor is~$u$).

\begin{lemma}[Private Neighbor Bound]\label{lem:pnb}
  For any tree~$T$ on $n$~vertices and any dominating independent set~$S$ of size~$k$, the total number of private vertices $P$ satisfies $P \ge n - 2k + 1$.
\end{lemma}

\begin{proof}
Since $S$ is independent, all edges from $S$ go to $V \setminus S$.
Let $Q = n - k - P$ be the number of non-$S$ vertices with $\ge 2$ $S$-neighbors.
Counting edges from~$S$: $|E(S, V \setminus S)| \ge P + 2Q = 2(n - k) - P$.
Since $T$ has $n - 1$ edges total, $P \ge 2(n - k) - (n - 1) = n - 2k + 1$.
\end{proof}

If $S$ is 1-Private, then $P \le k$, so $k \ge \lceil(n+1)/3\rceil = \lfloor n/3 \rfloor + 1$.
Thus every 1-Private maximal IS has $|S| \ge \lfloor n/3 \rfloor + 1$, and the Private Neighbor Property~-- that every maximal IS of size $k < \mathrm{mode}(I(T))$ has some $u$ with $\mathrm{priv}(u) \ge 2$~-- holds whenever $\mathrm{mode} \le \lfloor n/3 \rfloor + 1$.

For trees with higher modes, write $d_{\mathrm{leaf}}(v)$ for the number of leaf-children of~$v$.

\begin{lemma}[Hub Exclusion]\label{lem:hub-excl}
  If $S$ is a 1-Private maximal IS and $d_{\mathrm{leaf}}(v) \ge 2$, then $v \notin S$ and all leaf-children of~$v$ lie in~$S$.
\end{lemma}

\begin{proof}
If $v \in S$, every leaf-child $w$ satisfies $N(w) \cap S = \{v\}$, making $w$ private to~$v$; since $d_{\mathrm{leaf}}(v) \ge 2$, this gives $\mathrm{priv}(v) \ge 2$, contradicting 1-Private.
Since $v \notin S$, each leaf-child (with only~$v$ as a neighbor) must be in~$S$ for domination.
\end{proof}

\begin{lemma}[Transfer]\label{lem:transfer}
  Let $S$ be a 1-Private maximal IS in a tree~$T$ with a vertex~$v$ having $d_{\mathrm{leaf}}(v) = d \ge 2$ leaf-children $w_1, \dotsc, w_d$.
  Let $T' = T - \{v, w_1, \dotsc, w_d\}$.
  Then $S' = S \cap V(T')$ is a 1-Private maximal IS in~$T'$.
\end{lemma}

\begin{proof}
\emph{Maximality:} For $u \in V(T') \setminus S$, adding~$u$ to~$S$ in~$T$ creates a conflict with some $s \in S$.
If $s = w_i$, then $u$ is adjacent to~$w_i$, forcing $u = v$ (since $\deg(w_i) = 1$); but $v \notin V(T')$.
So $s \in S'$, and $u$ conflicts within~$T'$.

\emph{1-Private transfers:} The vertex~$v$ has $d \ge 2$ $S$-neighbors (the leaf-children in~$S$), so $v$ is not private to any $u \in S'$.
Each~$w_i$ has $N(w_i) = \{v\}$ and $v \notin S$, so $w_i$ is not private to any $u \in S'$ either.
Thus $\mathrm{priv}_{T'}(u) \subseteq \mathrm{priv}_T(u)$ and $|\mathrm{priv}_{T'}(u)| \le 1$.
\end{proof}

Hub Exclusion forces $|S| \ge d + |S'|$.
Applying the Private Neighbor Bound to each component of~$T'$ gives $|S'| \ge \sum_j \lceil(|C_j|+1)/3\rceil$.
Iterating~-- removing all vertices with $d_{\mathrm{leaf}} \ge 2$ and their leaf-children~-- terminates at a residual forest in which every vertex has $d_{\mathrm{leaf}} \le 1$.
The entire proof of unimodality thus reduces to a single conjecture about these residual trees.

\begin{conjecture}\label{conj:A}
  If every vertex of a tree~$T$ has $d_{\mathrm{leaf}}(v) \le 1$, then $\mathrm{mode}(I(T)) \le \lfloor n/3 \rfloor + 1$.
\end{conjecture}

I verified Conjecture~\ref{conj:A} for all 227\,678 $d_{\mathrm{leaf}} \le 1$ trees through $n = 22$ (zero violations).
The complementary bound for Case~B~-- $k \ge F + \lceil(n - F - h + c)/3\rceil$ where $F$ counts forced leaves, $h$ counts multi-leaf hubs, and $c$ counts residual components~-- was verified for all 8\,710\,881 trees with $d_{\mathrm{leaf}} \ge 2$ through $n = 22$ (zero violations, minimum surplus~1).

\paragraph*{Mean bound approach.}
In the hard-core model at fugacity $\lambda = 1$, the mean IS size is $\mu = \sum_v P(v \in S)$ where $P$ is the uniform measure over independent sets.
The tree recursion $R(v) = \prod_c 1/(1 + R(c))$ gives $P(v) = R(v)/(1 + R(v))$.
For any support vertex~$v$ with a leaf-child, the leaf contributes a factor $1/2$ to~$R(v)$, so $R(v) \le 1/2$ and $P(v) \le 1/3$.
Pairing each leaf with its support gives combined load $P(w) + P(v) \le 2/3$, exactly matching the $n/3$ budget.
The remaining core vertices can individually exceed $P(v) = 1/3$ (up to~${\approx}\,0.43$), but the accumulated slack from leaf-support pairs compensates; core path vertices relax toward the infinite-path limit of $P \approx 0.276$.
The mean satisfies $\mu < n/3$ for all 43\,029 $d_{\mathrm{leaf}} \le 1$ trees through $n = 20$ (max $\mu/(n/3) = 0.973$); the extremal family $S(2^k, 1)$ has gap $n/3 - \mu \to 1/6$.

\subsection*{Remaining directions}

\begin{itemize}
  \item Prove Conjecture~\ref{conj:A}.
    The hard-core decomposition into leaf-support pairs ($P(v) \le 1/3$ proved) and core vertices provides a candidate proof structure; bounding the core load is the remaining step.
  \item Establish an injection from level~$k$ to level~$k+1$ for $k$ below the mode.
    An augmented bipartite graph (containment plus swap edges) satisfies Hall's condition through $n = 18$ (204\,909 trees, zero failures), but a general proof remains open.
  \item A closed-form characterization of the optimal multi-arm configuration as a function of~$n$.
  \item Extension of the exhaustive search to $n = 27$ (751 million trees).
\end{itemize}

All code, data, and reproduction scripts are available at
\href{https://github.com/BrettRey/erdos-problem-993}{github.com/BrettRey/\allowbreak erdos-problem-993}.


\section*{Acknowledgements}

Computational exploration and code development were assisted by Claude Opus~4.6 (Anthropic) and Gemini~2.5~Pro (Google).

\appendix

\section{Reproducibility}\label{app:repro}

\paragraph*{Hardware.}
All computations were run on an Apple M4 (arm64) with 32\,GB RAM, running macOS.

\paragraph*{Software versions.}
Python~3.14.2, NumPy~2.4.2, NetworkX~3.6.1, nauty/geng~2.9301 (32-bit).

\paragraph*{Reproduction commands.}
\begin{verbatim}
pip install networkx numpy
brew install nauty

# Unit tests (37 tests)
python3 -m unittest test_all.py -v

# Exhaustive search, n <= 20 (single process)
python3 search.py --max-n 20

# Exhaustive search, n <= 26 (8 workers, requires geng)
python3 search.py --max-n 26 --workers 8

# Exhaustive n=26 analysis (LC + near-miss metrics)
python3 analyze.py 26 --workers 8 --top-k 200

# Targeted family search (n up to 500)
python3 targeted.py --max-n 500 --random-count 5000

# Broom asymptotic study
python3 broom_asymptotic.py

# Evolutionary nm optimizer (multi-arm star search)
python3 nm_optimizer.py --min-n 50 --max-n 200

# Finite-core enumeration (degree-2-free, leaf-light)
python3 scripts/finite_core_enum.py --b0 8 --leaf-cap 4 --out results/finite_core_b8_l4.json
\end{verbatim}

Random seeds are fixed (\texttt{random.Random(42)}) in \texttt{targeted.py} for caterpillar and Ramos--Sun families, ensuring deterministic reproduction.
Per-family breakdowns (counts, LC failures, best near-miss) are archived in \texttt{results/targeted\_families.json}.

\paragraph*{Data files.}
\begin{itemize}
  \item \texttt{results/analysis\_n26.json} -- Full n=26 re-analysis: 2 LC failures, top 200 near-misses with polynomials.
  \item \texttt{results/targeted\_n500.json} -- Top 500 near-misses from targeted search.
  \item \texttt{results/targeted\_families.json} -- Per-family summary (Table~\ref{tab:targeted}).
  \item \texttt{results/multi\_arm\_optimization.json} -- Multi-arm star vs.\ broom comparison (Table~\ref{tab:multiarm}).
  \item \texttt{results/finite\_core\_b*\_l*.json} -- Finite-core enumeration checks (degree-2-free, bounded leaf-load).
\end{itemize}


\section{Verification checklist}\label{app:verify}

\begin{enumerate}
  \item \textbf{Tree counts match OEIS A000055.}
    Every per-$n$ count reported by \texttt{search.py} (for $n \le 26$) was checked against A000055 \citep{oeis-a000055}.
    The search halts with an error if counts disagree.

  \item \textbf{Small-$n$ cross-check across backends.}
    For $n \le 10$ (tests), the independence polynomials computed via the geng pipeline and via NetworkX tree generation match exactly when geng is available.

  \item \textbf{Known examples verified.}
    The two LC-failing trees at $n = 26$ from \citet{kadrawi2023} were independently reproduced: our search found exactly two LC failures, both at index $k = 13$, with LC ratios 1.145 and 1.030.

  \item \textbf{Integer arithmetic for log-concavity.}
    All LC checks use $i_{k-1}\, i_{k+1}$ vs.\ $i_k^2$ in exact integer arithmetic (Python arbitrary-precision integers), with no floating-point rounding.

  \item \textbf{Overflow guard for polynomial multiplication.}
    \texttt{numpy.convolve} is used only when the maximum product of coefficient sizes fits in 64-bit integers (threshold: $\max(\mathit{terms}) \cdot \max(a) \cdot \max(b) < 2^{62}$).
    Beyond this, multiplication falls back to pure Python big-integer arithmetic.
    The guard triggers for tree families at $n \gtrsim 90$.

  \item \textbf{Unit tests.}
    37 tests cover: independence polynomial correctness on small graphs (paths, stars, known examples), unimodality and LC checks, near-miss ratio computation, graph6 parsing, and tree generation.
\end{enumerate}

\newpage
\printbibliography

\end{document}
