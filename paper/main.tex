% !TEX TS-program = xelatex
\documentclass[12pt,oneside]{article}
\input{preamble.tex}

% ============================
% Additional packages
% ============================
\usepackage{amsthm}

% Theorem environments
\newtheorem{theorem}{Theorem}
\newtheorem{lemma}{Lemma}
\newtheorem{corollary}{Corollary}
\newtheorem{conjecture}{Conjecture}
\newtheorem{definition}{Definition}
\theoremstyle{remark}
\newtheorem{remark}{Remark}

% ============================
% Notation macros
% ============================
\newcommand{\indpoly}{I(T;\,x)}           % independence polynomial
\newcommand{\ik}{i_k(T)}                  % coefficient
\newcommand{\nm}{\mathrm{nm}}             % near-miss ratio
\newcommand{\broom}{\mathrm{B}}           % broom notation

% ============================
% Metadata
% ============================
\title{Unimodality of independence polynomials of trees:\\
  computational verification and broom asymptotics}
\author{Brett Reynolds\,\orcidlink{0000-0003-2407-9448}}
\date{}

\begin{document}
\maketitle

\begin{abstract}
\citet{alavi1987} conjectured that the independence polynomial of every tree is unimodal.
I report three computational results bearing on this conjecture and an elementary analytic result for brooms.
These correspond to verified results within a finite range, adversarial stress-tests on structured families, and a heuristic asymptotic model.
First, I verify it exhaustively for all 447\,672\,596 non-isomorphic trees on at most 26 vertices, extending the previously published bound of $n \le 25$.
At $n = 26$ I confirm exactly two log-concavity failures and no unimodality failures, with every tree's coefficient sequence comfortably unimodal.
Second, I test 145\,362 trees from five structured families (subdivided stars, caterpillars, spiders, brooms, and random perturbations) at sizes up to $n = 500$, finding zero unimodality violations.
Third, I identify broom trees as the family approaching the unimodality boundary most closely and establish an empirical scaling law: for the broom $\broom(p,s)$, the near-miss ratio satisfies $\nm(s) = 1 - C/s + O(1/s^2)$ with $C \approx 4.12$, so that the margin of unimodality shrinks as $O(1/s)$ but never vanishes. The same asymptotic mechanism applies to any fixed core with $s$ pendant leaves at a single vertex.
I also give an elementary proof that brooms are unimodal whenever the star size dominates the path ($s \ge p$), and note that a recent theorem of Li, Li, Yang, and Zhang implies log-concavity (hence unimodality) for all brooms.\citep{li2025spiders}
\end{abstract}

\section{Introduction}\label{sec:intro}

An \emph{independent set} in a graph $G$ is a set of vertices no two of which are adjacent.
Write $i_k(G)$ for the number of independent sets of size~$k$ in~$G$, and let $\alpha(G)$ denote the independence number (the maximum size of an independent set).
The \emph{independence polynomial} of~$G$ is
\[
  I(G;\, x) \;=\; \sum_{k=0}^{\alpha(G)} i_k(G)\, x^k,
\]
with the convention $i_0(G) = 1$.

\begin{conjecture}[\citealp{alavi1987}]\label{conj:main}
  For every tree~$T$, the sequence $i_0(T), i_1(T), \dotsc, i_{\alpha(T)}(T)$ is unimodal.
\end{conjecture}

\citet{alavi1987} showed that the independence sequence of a general graph can realize any prescribed shape, but conjectured that trees are constrained enough to force unimodality.
The conjecture naturally suggested a stronger form: that the sequence should be \emph{log-concave}, meaning $i_k^2 \ge i_{k-1}\, i_{k+1}$ for all~$k$.
This stronger property held for all trees tested computationally until $n = 26$, when \citet{kadrawi2023} found exactly two trees whose independence sequences are unimodal but not log-concave.
\citet{galvin2025} subsequently constructed infinite families of trees (subdivided stars) with log-concavity failures arbitrarily far from the ends of the sequence.
\citet{ramos2025} used machine learning (PatternBoost) to find tens of thousands of log-concavity failures up to $n = 101$, but no unimodality failure.

On the positive side, unimodality has been proved for several tree families: paths, centipedes, regular caterpillars, and Fibonacci trees (which are real-rooted, hence log-concave), as well as well-covered spiders and certain restricted caterpillars.
\citet{levit2006} showed that $i_k(T)$ is strictly decreasing for $k \ge \lceil (2\alpha - 1)/3 \rceil$, so any unimodality violation must occur in roughly the first two-thirds of the sequence.
\citet{heilman2025} proved that for a uniformly random labelled tree on $n$ vertices, the first $\approx 46.8\%$ of the independence sequence is increasing with high probability; combined with the Levit--Mandrescu tail bound, roughly 85\% of the sequence is known to behave as expected almost surely.

The best previously published exhaustive verification is $n \le 25$ (A.~Radcliffe, personal communication, cited in several papers).
I extend this to $n \le 26$, test structured families up to $n = 500$, and identify brooms as the trees closest to violating unimodality.

My contributions:
\begin{enumerate}
  \item Exhaustive verification of Conjecture~\ref{conj:main} for all trees on $n \le 26$ vertices (Section~\ref{sec:exhaustive}).
  \item Targeted verification across five tree families at sizes up to $n = 500$ (Section~\ref{sec:targeted}).
  \item An empirical asymptotic analysis showing that broom trees---and more generally fixed-core leaf attachments---approach the unimodality boundary at rate $O(1/s)$ (Section~\ref{sec:broom}).
  \item An elementary proof that brooms are unimodal whenever $s \ge p$ (Section~\ref{sec:broom}).
\end{enumerate}
Since this work was completed, Li, Li, Yang, and Zhang proved that all spiders are strongly log-concave, which implies log-concavity (hence unimodality) for brooms as a special case.\citep{li2025spiders}
I include the elementary $s \ge p$ proof as a self-contained argument that aligns with the asymptotic analysis.
These results should be read as three distinct evidence types: exhaustive verification provides certified confirmation within the tested range; targeted family searches are adversarial stress-tests; and the broom analysis offers an explanatory asymptotic model rather than a general proof.


\section{Definitions and method}\label{sec:method}

\begin{definition}
  A finite sequence $(a_0, a_1, \dotsc, a_m)$ of positive reals is \emph{unimodal} if there exists an index~$p$ such that $a_0 \le a_1 \le \dotsb \le a_p \ge a_{p+1} \ge \dotsb \ge a_m$.
  It is \emph{log-concave} if $a_k^2 \ge a_{k-1}\, a_{k+1}$ for all $1 \le k \le m-1$.
\end{definition}

Log-concavity (with positive terms) implies unimodality, but not conversely.

\paragraph*{Near-miss ratio.}
To quantify how close a tree comes to violating unimodality, I define the \emph{near-miss ratio}.
Let $j_0$ be the first index where $i_{j_0} > i_{j_0+1}$ (the first strict descent).
The near-miss ratio is
\[
  \nm(T) \;=\; \max_{j > j_0} \frac{i_{j+1}(T)}{i_j(T)}.
\]
If $\nm(T) > 1$, the sequence rises after descending, violating unimodality.
A value near 1 indicates that the tree nearly violates it.
Thus $\nm$ serves as a quantitative measure of how tight the conjecture is for a family.

\paragraph*{Tree DP.}
The computation of $I(T;\, x)$ proceeds by rooting $T$ at an arbitrary vertex and performing a bottom-up traversal.
For each vertex~$v$, two polynomials are maintained: $d_v^{(0)}(x)$ counts independent sets in the subtree of~$v$ that exclude~$v$; $d_v^{(1)}(x)$ counts those that include~$v$.
The recurrences are
\begin{align}
  d_v^{(0)} &= \prod_{c \,\in\, \mathrm{children}(v)} \bigl(d_c^{(0)} + d_c^{(1)}\bigr), \label{eq:dp0}\\
  d_v^{(1)} &= x \cdot \prod_{c \,\in\, \mathrm{children}(v)} d_c^{(0)}, \label{eq:dp1}
\end{align}
and $I(T;\, x) = d_r^{(0)} + d_r^{(1)}$ for the root~$r$.
The total time per tree is $O(n^2)$, dominated by polynomial multiplications via naive convolution.

\paragraph*{Computational setup.}
Trees are enumerated with \texttt{nauty/geng} \citep{mckay2014}, which generates all non-isomorphic connected graphs on $n$ vertices with exactly $n - 1$ edges.
Computation is parallelized by passing \texttt{geng}'s \texttt{res/mod} partitioning to eight workers via Python's \texttt{multiprocessing}.
Polynomial arithmetic uses \texttt{numpy.convolve} when coefficients fit in 64-bit integers, with automatic fallback to pure Python big-integer arithmetic for larger values (relevant at $n \gtrsim 90$).
Log-concavity is checked using integer arithmetic ($i_k^2$ vs.\ $i_{k-1}\, i_{k+1}$), avoiding floating-point error.


\section{Exhaustive verification through $n = 26$}\label{sec:exhaustive}

Table~\ref{tab:exhaustive} shows the verification results.
Every tree count matches OEIS A000055 \citep{oeis-a000055}.
No unimodality violation was found at any~$n$.
Counts are checked automatically against A000055 during the search; the run halts on any mismatch.

\begin{table}[ht]
  \centering
  \caption{Exhaustive verification of unimodality for trees on $n$ vertices.  Times are for an Apple Silicon Mac with 8 parallel workers ($n \ge 21$) or a single process ($n \le 20$).}
  \label{tab:exhaustive}
  \begin{tabular}{r r r}
    \toprule
    $n$ & Trees & Time \\
    \midrule
    1--15 & \liningnums{13\,188} & ${<}\,1$\,s \\
    16 & \liningnums{19\,320} & 1\,s \\
    17 & \liningnums{48\,629} & 3\,s \\
    18 & \liningnums{123\,867} & 9\,s \\
    19 & \liningnums{317\,955} & 23\,s \\
    20 & \liningnums{823\,065} & 68\,s \\
    21 & \liningnums{2\,144\,505} & 55\,s \\
    22 & \liningnums{5\,623\,756} & 1\,m\,44\,s \\
    23 & \liningnums{14\,828\,074} & 4\,m\,41\,s \\
    24 & \liningnums{39\,299\,897} & 12\,m\,5\,s \\
    25 & \liningnums{104\,636\,890} & 38\,m\,33\,s \\
    26 & \liningnums{279\,793\,450} & 4\,h\,51\,m \\
    \midrule
    Total & \liningnums{447\,672\,596} & \\
    \bottomrule
  \end{tabular}
\end{table}

The case $n = 26$ is particularly interesting because it is the first~$n$ at which log-concavity fails \citep{kadrawi2023}.
A re-analysis of all 279\,793\,450 trees at $n = 26$ found:
\begin{itemize}
  \item Exactly 2 log-concavity failures (consistent with \citealp{kadrawi2023}), both at index $k = 13$.
    The worst log-concavity ratio is $i_{12}\, i_{14} / i_{13}^2 = 1.145$, exceeding the threshold of~1.
  \item The best (highest) near-miss ratio at $n = 26$ is $\nm = 0.845$, far below the violation threshold of~1.
\end{itemize}

In other words, even at the first~$n$ where the stronger log-concavity property breaks, unimodality holds with a comfortable margin: the tightest near-miss is only 84.5\% of the way to a violation.


\section{Targeted search on structured families}\label{sec:targeted}

Beyond exhaustive enumeration, I tested 145\,362 trees from five families chosen for their proximity to the unimodality boundary.
Results appear in Table~\ref{tab:targeted}.
This is an adversarial stress-test rather than a proof: the absence of violations only rules out counterexamples within these families and size ranges.

\begin{table}[ht]
  \centering
  \caption{Targeted verification on structured tree families ($n$ up to 500).
    LC = log-concavity.  Best~nm is the highest near-miss ratio observed in the family.}
  \label{tab:targeted}
  \begin{tabular}{l r r l}
    \toprule
    Family & Trees & LC failures & Best nm \\
    \midrule
    Galvin SST $T_{m,t,1}$     & \liningnums{571}     & \liningnums{108} & 0.936 \\
    Generalized SST $T_{m,t,d}$ & \liningnums{680}    & \liningnums{268} & 0.981 \\
    Caterpillars                & \liningnums{5\,196}  & 0 & -- \\
    Spiders and brooms          & \liningnums{133\,915}& 0 & 0.992 \\
    Random (Ramos--Sun style)   & \liningnums{5\,000}  & \liningnums{2} & 0.804 \\
    \midrule
    Total                       & \liningnums{145\,362}& \liningnums{378} &  \\
    \bottomrule
  \end{tabular}
\end{table}

A complementary structural scan considered the class
\[
  \mathcal{C}_2 = \{T:\ |\{v \in V(T): \deg(v)\ge 3\}| \le 2\},
\]
which strictly contains spiders.
Exhaustive filtering of all non-isomorphic trees through $n=24$ gave 196\,635 trees in $\mathcal{C}_2$ and found zero log-concavity failures (hence zero unimodality failures).
Equivalently: all trees with at most two branch vertices up to $n=24$ are log-concave (hence unimodal).
The worst observed log-concavity ratio was $0.846153846\ldots$, still well below the violation threshold of~1.
Full run details are archived in \texttt{results/two\_branch\_lc\_n24.json}.

\begin{table}[ht]
  \centering
  \caption{Compact summary of the $\mathcal{C}_2$ scan for $n=20,\dots,24$.}
  \label{tab:c2scan}
  \begin{tabular}{r r r r}
    \toprule
    $n$ & Trees & $\mathcal{C}_2$ trees & LC failures in $\mathcal{C}_2$ \\
    \midrule
    20 & \liningnums{823\,065}    & \liningnums{13\,463} & 0 \\
    21 & \liningnums{2\,144\,505} & \liningnums{20\,267} & 0 \\
    22 & \liningnums{5\,623\,756} & \liningnums{30\,170} & 0 \\
    23 & \liningnums{14\,828\,074}& \liningnums{44\,385} & 0 \\
    24 & \liningnums{39\,299\,897}& \liningnums{64\,674} & 0 \\
    \bottomrule
  \end{tabular}
\end{table}

No unimodality violation was found in any family.
Three patterns stand out:
\begin{itemize}
  \item \textbf{Subdivided stars dominate log-concavity failures} (376 of 378), consistent with \citet{galvin2025}, but their near-miss ratios stay well below~1.
  \item \textbf{Caterpillars are perfectly log-concave} through $n = 500$, with zero failures.
  \item \textbf{Brooms produce the highest near-miss ratios} ($\nm = 0.992$ at $n = 500$), surpassing all SST variants despite having no log-concavity failures.
\end{itemize}

A \emph{broom} $\broom(p, s)$ consists of a path on $p$ vertices with a star of $s$ leaves attached at one end; it has $n = p + s$ vertices and independence number $\alpha = \lfloor p/2 \rfloor + s$.
Brooms interpolate between path-like and star-like extremes, combining a long induced path with a high-degree hub, and are therefore natural candidates for near-boundary behavior.
The broom $\broom(33, 467)$ at $n = 500$ achieves $\nm = 0.9917$, the highest observed anywhere in this study.
This makes brooms, not subdivided stars, the family closest to the unimodality boundary.


\section{Broom asymptotics}\label{sec:broom}

The targeted results raised a natural question: does the broom near-miss ratio converge to~1 as $s \to \infty$?
If so, the conjecture would be true for brooms but only barely, with the margin vanishing in the limit.
I first fit an empirical scaling law and then justify it via a leading-term asymptotic analysis.
The same asymptotic mechanism applies more generally to any fixed core with $s$ pendant leaves at a single vertex.

\begin{theorem}[Broom unimodality for large stars]\label{thm:broom-sgeqp}
Fix $p \ge 2$. For all $s \ge p$, the broom $\broom(p,s)$ is unimodal.
\end{theorem}

\begin{proof}
Write $A(x) = I(P_{p-1}; x) = \sum_t a_t x^t$ and $B(x) = I(P_{p-2}; x) = \sum_t b_t x^t$.
Then
\[
  I(\broom(p,s); x) = (1+x)^s A(x) + x B(x).
\]
Let $c_k$ be the coefficient of $x^k$ in $I(\broom(p,s); x)$ and write $c_k = d_k + e_k$, where
\[
  d_k = \sum_{j=0}^{\min(s,k)} \binom{s}{j}\, a_{k-j}, \qquad e_k = b_{k-1}.
\]
Let $S = \lfloor (p-1)/2 \rfloor + 1$, so $e_k = 0$ for $k \ge S+1$.
Since $a_t$ and $\binom{s}{j}$ are log-concave with no internal zeros, $d_k$ is log-concave and hence unimodal.
It suffices to show $c_{k+1} \ge c_k$ for all $0 \le k \le S-1$.

For such $k$, change variables $t = k+1-j$ to obtain
\[
  d_{k+1} - d_k = \sum_t a_t\Bigl(\binom{s}{k+1-t} - \binom{s}{k-t}\Bigr).
\]
Because $k+1 \le S = \lceil p/2 \rceil \le \lceil s/2 \rceil$, and the binomial row is nondecreasing up to $r = \lceil s/2 \rceil$,
each term in parentheses is nonnegative.
Keeping only the $t=1$ term gives
\[
  d_{k+1} - d_k \ge a_1\Bigl(\binom{s}{k} - \binom{s}{k-1}\Bigr) = (p-1)\Bigl(\binom{s}{k} - \binom{s}{k-1}\Bigr).
\]
For fixed $k \le \lfloor p/2 \rfloor$, the function $s \mapsto \binom{s}{k} - \binom{s}{k-1}$ is increasing for $s \ge 2k-1$, hence
\[
  \binom{s}{k} - \binom{s}{k-1} \ge \binom{p}{k} - \binom{p}{k-1}.
\]
Using $\binom{p}{k} = \binom{p}{k-1}\frac{p-k+1}{k}$, this implies
\[
  d_{k+1} - d_k \ge (p-1)\binom{p}{k-1}\frac{p-2k+1}{k}.
\]
Since $k \le \lfloor p/2 \rfloor$, we have $p-2k+1 \ge 1$ and thus
$(p-1)\binom{p}{k-1}\frac{p-2k+1}{k} \ge \binom{p}{k-1}$.
Finally, $e_k - e_{k+1} = b_{k-1} - b_k \le b_{k-1} = \binom{p-k}{k-1} \le \binom{p}{k-1}$.
Therefore $d_{k+1} - d_k \ge e_k - e_{k+1}$, so $c_{k+1} \ge c_k$ for $0 \le k \le S-1$.

For $k \ge S$, we have $c_k = d_k$, which is unimodal. Hence $c_k$ is unimodal for all $s \ge p$.
\end{proof}

\begin{corollary}[From the spider theorem]\label{cor:broom-all}
For all $p \ge 2$ and $s \ge 0$, the broom $\broom(p,s)$ has a log-concave (hence unimodal) independence sequence.
\end{corollary}
\begin{proof}
Log-concavity with positive terms implies unimodality: if $a_k^2 \ge a_{k-1}a_{k+1}$ for all $k$, then the ratios
$r_k = a_k/a_{k-1}$ are nonincreasing, so the sequence increases up to a peak and then decreases.

If $s \le 1$, the broom is a path ($\broom(p,0)=P_p$ and $\broom(p,1)=P_{p+1}$), whose independence polynomial is real-rooted and hence log-concave.
If $s \ge 2$, the broom is a spider (a unique vertex of degree at least~3, with one long leg and $s$ legs of length~1).
Li, Li, Yang, and Zhang prove that every spider has a \emph{strongly log-concave} independence polynomial (their Theorem~3.1), which in particular implies $a_k^2 \ge a_{k-1}a_{k+1}$ by taking the adjacent-index specialization.\citep{li2025spiders}
Therefore $\broom(p,s)$ is log-concave for all $s \ge 0$, and hence unimodal.
\end{proof}

I computed $\broom(13, s)$ for $s$ up to 20\,000 (trees with over 20\,000 vertices, independence polynomials with degrees in the tens of thousands).
Table~\ref{tab:broom} shows the results.

\begin{table}[ht]
  \centering
  \caption{Broom $\broom(13, s)$: convergence of the near-miss ratio.
    The scaled gap $s \cdot (1 - \nm)$ stabilizes near $C \approx 4.12$.}
  \label{tab:broom}
  \begin{tabular}{r r l l l}
    \toprule
    $s$ & $n$ & $\nm$ & $1 - \nm$ & $s \cdot (1 - \nm)$ \\
    \midrule
    1\,000  & 1\,013  & 0.995\,911 & 0.004\,09 & 4.089 \\
    2\,000  & 2\,013  & 0.997\,947 & 0.002\,05 & 4.105 \\
    5\,000  & 5\,013  & 0.999\,177 & 0.000\,82 & 4.115 \\
    10\,000 & 10\,013 & 0.999\,588 & 0.000\,41 & 4.119 \\
    20\,000 & 20\,013 & 0.999\,794 & 0.000\,21 & 4.120 \\
    \bottomrule
  \end{tabular}
\end{table}

The data fit the scaling law
\begin{equation}\label{eq:scaling}
  \nm(s) \;=\; 1 - \frac{C}{s} + O(1/s^2),
\end{equation}
with $C \approx 4.12$ for $p = 13$.
The same qualitative behaviour was observed (convergence of $s \cdot (1 - \nm)$ to a constant) for path lengths $p \in \{13, 22, 33, 42, 50\}$, with the constant~$C$ varying by path length.
In every case, $\nm < 1$ for all tested~$s$, so all brooms remain unimodal.

\paragraph*{Heuristic for the $1 - C/s$ law.}
More generally, let $H$ be any fixed tree with distinguished vertex $v$, and let $H_s$ be obtained
by attaching $s$ new leaves to $v$. The vertex-deletion recurrence gives
\[
  I(H_s; x) = (1+x)^s\, A(x) + x\, B(x),
  \quad A(x) = I(H - v; x), \quad B(x) = I(H - N[v]; x).
\]

For brooms, let $\broom(p, s)$ be a path on $p$ vertices with $s$ leaves attached at one end.
Applying the same recurrence at the attachment vertex gives the closed form
\begin{equation}\label{eq:broom-closed}
  I(\broom(p, s); x) \;=\; (1+x)^s\, I(P_{p-1}; x) \;+\; x\, I(P_{p-2}; x),
\end{equation}
where $P_m$ is the path on $m$ vertices.
Write $A(x) = I(P_{p-1}; x) = \sum_{j=0}^{d} a_j x^j$.
The quantities $A(1)$ and $A'(1)$ that govern the scaling constant are computable from a simple recurrence.
Since $I(P_n; x) = I(P_{n-1}; x) + x\, I(P_{n-2}; x)$, setting $F_n = I(P_n; 1)$ and $G_n = I'(P_n; 1)$ gives
\begin{align}
  F_n &= F_{n-1} + F_{n-2}, \label{eq:Fn}\\
  G_n &= G_{n-1} + G_{n-2} + F_{n-2}, \label{eq:Gn}
\end{align}
with $F_0 = 1$, $F_1 = 2$, $G_0 = 0$, $G_1 = 1$.
In particular, $F_n$ is the $(n+2)$-nd Fibonacci number, and $A(1) = F_{p-1}$, $A'(1) = G_{p-1}$.

For large $s$, the $x\, I(P_{p-2}; x)$ term in~\eqref{eq:broom-closed} is negligible in the central coefficient window: its contribution to $c_k$ is bounded by $I(P_{p-2}; 1)$, while the leading term contributes $\Theta(\binom{s}{k}) \cdot A(1)$, which is exponentially larger near $k \approx s/2$.
The omission perturbs $r_k = c_{k+1}/c_k$ only at $O(1/s^2)$.

Dropping this term, for $k \approx s/2$ we have
\[
  c_k \;\approx\; \sum_{j=0}^{d} a_j \binom{s}{k-j}.
\]
For fixed $j$ and $k = s/2 + y$ with $y = O(1)$,
\[
  \frac{\binom{s}{k+1-j}}{\binom{s}{k-j}}
  \;=\; \frac{s-k+j}{k+1-j}
  \;=\; 1 - \frac{4y + 2 - 4j}{s} + O(s^{-2}).
\]
Taking the $a_j$-weighted average yields
\[
  r_k \;=\; 1 - \frac{4y + 2 - 4\mu}{s} + O(s^{-2}),
  \quad \mu = \frac{A'(1)}{A(1)} = \frac{G_{p-1}}{F_{p-1}}.
\]
The first descent occurs near $y_0 \approx \mu - \tfrac{1}{2}$.
Because the near-miss ratio scans indices strictly after the first descent, the
maximal ratio occurs at the smallest integer $m \ge \mu + \tfrac{1}{2}$, giving
\begin{equation}\label{eq:Cp}
  \nm(s) \;\approx\; 1 - \frac{C(p)}{s}, \quad
  C(p) = (4m + 2) - 4\mu.
\end{equation}
Table~\ref{tab:Cpred} compares the predicted $C(p)$ from~\eqref{eq:Cp} against the empirical value at $s = 5{,}000$ for each tested path length.

\begin{table}[ht]
  \centering
  \caption{Predicted vs.\ observed scaling constant $C(p)$ for broom $\broom(p, s)$.
    $C_{\mathrm{pred}}$ is computed from the Fibonacci recurrences~\eqref{eq:Fn}--\eqref{eq:Gn};
    $C_{\mathrm{obs}}$ is $s \cdot (1 - \nm)$ at $s = 5{,}000$.}
  \label{tab:Cpred}
  \begin{tabular}{r r r l l l}
    \toprule
    $p$ & $F_{p-1}$ & $G_{p-1}$ & $\mu$ & $C_{\mathrm{pred}}$ & $C_{\mathrm{obs}}$ \\
    \midrule
    13 & \liningnums{377} & \liningnums{1\,308} & 3.470 & 4.122 & 4.115 \\
    22 & \liningnums{28\,657} & \liningnums{170\,711} & 5.957 & 6.172 & 6.156 \\
    33 & \liningnums{5\,702\,887} & \liningnums{51\,310\,978} & 8.997 & 6.011 & 5.991 \\
    42 & \liningnums{433\,494\,437} & \liningnums{4\,978\,643\,596} & 11.485 & 4.060 & 4.045 \\
    50 & \liningnums{20\,365\,011\,074} & \liningnums{278\,920\,277\,425} & 13.696 & 7.216 & 7.182 \\
    \bottomrule
  \end{tabular}
\end{table}

The agreement is within 0.5\% in all cases, confirming that the heuristic captures the leading behaviour.

\begin{theorem}[Asymptotic leaf-attachment unimodality]\label{thm:leaf-attach-asymptotic}
Let $H$ be a fixed tree with distinguished vertex $v$, and let $H_s$ be obtained by attaching
$s$ new leaves to $v$. Write $A(x) = I(H - v; x)$ with
$\mu = A'(1)/A(1)$ and $m = \lceil \mu + \tfrac{1}{2} \rceil$.
Let $\nm(s)$ be the near-miss ratio of $H_s$.
Then, as $s \to \infty$,
\[
  \nm(s) \;=\; 1 - \frac{C}{s} + O\!\left(\frac{1}{s^2}\right),
  \quad C = (4m + 2) - 4\mu.
\]
In particular, $C \in [4, 8)$, so $\nm(s) < 1$ for all sufficiently large $s$ and $H_s$ is unimodal for all $s \ge s_0(H)$.
\end{theorem}

\begin{proof}
Let $B(x) = I(H - N[v]; x)$ and write
$I(H_s; x) = (1+x)^s A(x) + x B(x)$.
Let $c_k = [x^k] I(H_s; x)$ and write $c_k = d_k + e_k$ with
\[
  d_k = \sum_{j=0}^{d} a_j \binom{s}{k-j}, \qquad
  e_k = [x^k]\, x B(x),
\]
where $A(x)=\sum_{j=0}^d a_j x^j$ and $d = \deg A$ depends only on $H$.
Define $r_k = d_{k+1}/d_k$.
Since
\[
  r_k
  = \frac{\sum_{j=0}^d a_j \binom{s}{k-j} \,\frac{\binom{s}{k+1-j}}{\binom{s}{k-j}}}
         {\sum_{j=0}^d a_j \binom{s}{k-j}},
\]
$r_k$ is a weighted average of the ratios
$\rho_j(k) = \binom{s}{k+1-j}/\binom{s}{k-j} = (s-k+j)/(k+1-j)$.
As $j$ increases, $\rho_j(k)$ increases, so $r_k \in [\rho_0(k), \rho_d(k)]$.
In particular, for $k \le \lfloor s/2 \rfloor - d - 1$ we have $\rho_0(k) > 1$, hence $r_k > 1$,
and for $k \ge \lfloor s/2 \rfloor + d$ we have $\rho_d(k) < 1$, hence $r_k < 1$.
Thus the first descent occurs for $k = \lfloor s/2 \rfloor + y$ with $|y| \le d+1$.

For such $k$, each $\rho_j(k)$ admits the uniform expansion
\[
  \rho_j(k) = 1 - \frac{4y + 2 - 4j}{s} + O\!\left(\frac{1}{s^2}\right),
\]
since $j$ and $y$ are $O_H(1)$. Taking the weighted average yields
\[
  r_k = 1 - \frac{4y + 2 - 4\mu}{s} + O\!\left(\frac{1}{s^2}\right),
  \qquad \mu = \frac{\sum_j j a_j}{\sum_j a_j} = \frac{A'(1)}{A(1)}.
\]
Moreover, in this central window we have $e_k/d_k = O(2^{-s})$ because
$e_k$ is $O_H(1)$ while $d_k \asymp \binom{s}{\lfloor s/2 \rfloor}$, so
$c_{k+1}/c_k = r_k + O(2^{-s})$.
Therefore $c_{k+1}/c_k$ decreases by $4/s + O(1/s^2)$ when $k$ increments by~1,
and the first descent occurs at $y_0 = \mu - \tfrac{1}{2} + O(1/s)$.
The maximal ratio after the descent is attained at the smallest integer
$m \ge \mu + \tfrac{1}{2}$, giving
\[
  \nm(s) = 1 - \frac{(4m + 2) - 4\mu}{s} + O\!\left(\frac{1}{s^2}\right).
\]
Since $m - (\mu + \tfrac{1}{2}) \in [0, 1)$, we have $C \in [4, 8)$, so
$\nm(s) < 1$ for all sufficiently large $s$.
\end{proof}

\begin{remark}
  The constant $C$ depends only on $\mu$ and its ceiling. For brooms,
  $\mu = G_{p-1}/F_{p-1}$ by the Fibonacci recurrences~\eqref{eq:Fn}--\eqref{eq:Gn},
  so $C(p)$ oscillates between 4 and 8 as $p$ varies.
  The theorem gives eventual unimodality for any fixed core; it does not
  address small $s$. For brooms, log-concavity (hence unimodality) for all
  $s$ follows from the spider theorem of Li et al.\citep{li2025spiders}
\end{remark}


\section{Discussion}\label{sec:discussion}

These results provide strong computational evidence for Conjecture~\ref{conj:main}.
Across 447.7 million exhaustively tested trees and 145 thousand structurally targeted trees, no unimodality violation was found.
The conjecture appears robust, but the broom asymptotic analysis reveals that it is also tight: the margin of unimodality can be made arbitrarily small.
Read as evidence types, the exhaustive portion certifies the conjecture within the tested range, the targeted families act as adversarial probes, and the broom analysis offers an explanatory asymptotic model for near-boundary behavior.
More generally, the fixed-core leaf-attachment theorem shows that this $1/s$ near-miss scaling is a generic effect of binomial smoothing, not a broom-specific quirk.

The surprise is that brooms, not subdivided stars, are closest to the boundary.
Subdivided stars produce many log-concavity failures but stay well within the unimodality threshold.
Brooms remain log-concave yet push the near-miss ratio toward~1.\citep{li2025spiders}
This suggests that the mechanisms behind log-concavity failure and near-unimodality violation are distinct.

Several directions remain:
\begin{itemize}
  \item An elementary proof of broom unimodality that avoids the spider theorem, exploiting the closed-form independence polynomial of broom trees.
  \item Extension of log-concavity proofs to larger structural classes (e.g., trees with at most two branch vertices).
  \item Extension of the exhaustive search to $n = 27$ (751 million trees). The current Python implementation would require roughly 12--15 hours on 8 cores; a compiled inner loop would reduce this substantially.
  \item Adaptation of PatternBoost \citep{ramos2025} to optimize for near-miss ratio rather than log-concavity failure, which might discover families even closer to the boundary.
\end{itemize}

All code, data, and reproduction scripts are available at \url{https://github.com/BrettRey/erdos-problem-993}.


\section*{Acknowledgements}

Computational exploration and code development were assisted by Claude Opus~4.6 (Anthropic).

\appendix

\section{Reproducibility}\label{app:repro}

\paragraph*{Hardware.}
All computations were run on an Apple M4 (arm64) with 32\,GB RAM, running macOS.

\paragraph*{Software versions.}
Python~3.14.2, NumPy~2.4.2, NetworkX~3.6.1, nauty/geng~2.9301 (32-bit).

\paragraph*{Reproduction commands.}
\begin{verbatim}
pip install networkx numpy
brew install nauty

# Unit tests (36 tests)
python3 -m unittest test_all.py -v

# Exhaustive search, n <= 20 (single process)
python3 search.py --max-n 20

# Exhaustive search, n <= 26 (8 workers, requires geng)
python3 search.py --max-n 26 --workers 8

# Exhaustive n=26 analysis (LC + near-miss metrics)
python3 analyze.py 26 --workers 8 --top-k 200

# Targeted family search (n up to 500)
python3 targeted.py --max-n 500 --random-count 5000

# Broom asymptotic study
python3 broom_asymptotic.py
\end{verbatim}

Random seeds are fixed (\texttt{random.Random(42)}) in \texttt{targeted.py} for caterpillar and Ramos--Sun families, ensuring deterministic reproduction.
Per-family breakdowns (counts, LC failures, best near-miss) are archived in \texttt{results/targeted\_families.json}.

\paragraph*{Data files.}
\begin{itemize}
  \item \texttt{results/analysis\_n26.json} -- Full n=26 re-analysis: 2 LC failures, top 200 near-misses with polynomials.
  \item \texttt{results/targeted\_n500.json} -- Top 500 near-misses from targeted search.
  \item \texttt{results/targeted\_families.json} -- Per-family summary (Table~\ref{tab:targeted}).
\end{itemize}


\section{Verification checklist}\label{app:verify}

\begin{enumerate}
  \item \textbf{Tree counts match OEIS A000055.}
    Every per-$n$ count reported by \texttt{search.py} (for $n \le 26$) was checked against A000055 \citep{oeis-a000055}.
    The search halts with an error if counts disagree.

  \item \textbf{Small-$n$ cross-check across backends.}
    For $n \le 10$ (tests), the independence polynomials computed via the geng pipeline and via NetworkX tree generation match exactly when geng is available.

  \item \textbf{Known examples verified.}
    The two LC-failing trees at $n = 26$ from \citet{kadrawi2023} were independently reproduced: our search found exactly two LC failures, both at index $k = 13$, with LC ratios 1.145 and 1.030.

  \item \textbf{Integer arithmetic for log-concavity.}
    All LC checks use $i_{k-1}\, i_{k+1}$ vs.\ $i_k^2$ in exact integer arithmetic (Python arbitrary-precision integers), with no floating-point rounding.

  \item \textbf{Overflow guard for polynomial multiplication.}
    \texttt{numpy.convolve} is used only when the maximum product of coefficient sizes fits in 64-bit integers (threshold: $\max(\mathit{terms}) \cdot \max(a) \cdot \max(b) < 2^{62}$).
    Beyond this, multiplication falls back to pure Python big-integer arithmetic.
    The guard triggers for tree families at $n \gtrsim 90$.

  \item \textbf{Unit tests.}
    36 tests cover: independence polynomial correctness on small graphs (paths, stars, known examples), unimodality and LC checks, near-miss ratio computation, graph6 parsing, and tree generation.
\end{enumerate}

\newpage
\printbibliography

\end{document}
